% rubber: module pdftex
% rubber: path tktl
% rubber: bibtex.stylepath tktl
% rubber: bibtex.path .

\documentclass{tktltiki}
\usepackage{ae,aecompl}
\usepackage{url}
\usepackage{amsfonts}
\usepackage{color}
\usepackage{graphicx}

\begin{document}
\title{Esimerkki}
\author{Jukka Suomela}
\date{\today}
\level{Tieteellisen kirjoittamisen kurssin tutkielma}
\maketitle

\onehalfspacing

\level{Tutkielma}
\faculty{Matemaattis-luonnontieteellinen}
\department{Tietojenk�sittelytieteen laitos}
\subject{Tietojenk�sittelytiede}
\numberofpagesinformation{\numberofpages\ sivua}
\classification{ \\
  F.1.3 [Computation by Abstract Devices]: Complexity Measures and Classes, \\
  F.2.2 [Analysis of Algorithms and Problem Complexity]: Nonnumerical Algorithms and Problems, \\
  F.4.1 [Mathematical Logic and Formal Languages]: Mathematical Logic, \\
  I.2.4 [Artificial Intelligence]: Knowledge Representation Formalisms and Methods
}
\keywords{esimerkki}

\begin{abstract}
    Tiivistelm�.
\end{abstract}

\mytableofcontents

\section{Johdanto}

T�m� on esimerkkilause \cite{haastad99clique}. Katso my�s kuva~\ref{fig:esimerkkikuva}.

\begin{figure}
    \centering
    \input{esimerkkikuva.pdf_t}
    \caption{Kuvateksti.}\label{fig:esimerkkikuva}
\end{figure}


\bibliographystyle{tktl}
\bibliography{lahteet}

\lastpage

\end{document}
